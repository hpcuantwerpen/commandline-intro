\enableregime[utf]

\environment env_vscfonts
\environment env_vscstyle

\useURL
  [coreutilsdocs]
  [https://www.gnu.org/software/coreutils/manual/html_node/index.html]
\useURL
  [bashmanual]
  [https://www.gnu.org/software/bash/manual/html_node/index.html]
\useURL
  [sedmanual]
  [https://www.gnu.org/software/sed/manual/html_node/index.html]

\starttext

\component c_titlepage

\placecombinedlist[content]

\chapter{Introduction}
This is a short reference, written for the course
\quotation{Introduction to Linux} for new users of the UAntwerpen HPC
clusters.  The purpose of this document is to explain some of the most
commonly used shell commands with their basic options in an accessible
way, so new users can quickly gain some practical experience.

These instructions should provide you with all the information you
need in order to solve the exercises, but only serve as an
introduction to the many options and features of these commands.  To
see all available options for a command, you can read the man page:
\Prompt{man [COMMAND]}

Man pages are often written in a very succinct style.  For more
elaborate explanations and detailed discussion of every feature of a
command, the manuals provided by the GNU project are very useful.

\startitemize
  \item Most commands discussed here are part of
a package called \quote{coreutils}, for which you can find the manual
at

\from[coreutilsdocs].

\item For information on Bash and built-in shell commands
  specifically, we refer to the Bash manual at

  \from[bashmanual].

\item \Command{sed} is an extremely versatile tool with a separate
  manual of its own:

  \from[sedmanual].
\stopitemize

On systems where these manuals are installed, you can read them from
the terminal as well.  To read the manual section on a command, typing
\Prompt{info [COMMAND]} usually takes you straight to the manual
section about that command.  For instructions on how to use the
\Command{info} viewer itself, you can run
\Prompt{info info "Getting Started" Help}

Note: there are many different shells (i.e.\ \Command{bash},
\Command{sh}, \Command{ksh}, \textellipsis), and many different
implementations of these command line tools (GNU, BSD, commercial
implementations, \textellipsis).  This document assumes we are using the
Bash shell and the GNU command line tools, which are the default on
the VSC systems and probably most HPC clusters worldwide.
Nevertheless, a lot of the material applies to all versions and
implementations.

\chapter{The file system}
\subsubject{\Command{ls} --- List directory contents}
\Prompt{ls [OPTION]... [FILE]...}

The \index[LS]{\Command{ls}}\Command{ls} command \index{directory+list
 contents of}lists files and directories on the file system.  If if
no \Arg{FILE} arguments are given, it lists the contents of the
current directory.  With \Arg{FILE} arguments, \Command{ls} will list
each file or --- if \Arg{FILE} is the name of a directory --- its contents.

Many options are available to make \Command{ls} display additional
information about files, or change the output order and formatting.
Some of the most commonly used options are

\Optiondesc{-a, --all} Do not ignore entries starting with \quote{\tt{.}}.

\Optiondesc{-d, --directory} List directories themselves, not their contents.

\Optiondesc{-l} Use a long listing format.  This displays additional
information such as file size, modification date and time, ownership
and permissions.

\Optiondesc{-h, --human-readable} If used together with \Arg{-l},
print file sizes in human readable format.

\Optiondesc{--sort} \index{sorting+\Command{ls}: sort output}By
default, \Command{ls} displays entries in alphabetical order.  You can
use the options \Arg{--sort=time} or \Arg{--sort=size} --- or the
respective short options \Arg{-t} and \Arg{-S} --- to sort by
modification time or file size.  Other sorting options are
\Arg{extension} (\Arg{-X}), \Arg{version} (\Arg{-v}), or \Arg{none}
(\Arg{-U})\footnote{With sorting order \Arg{none}, files are displayed
  in the order in which they are stored in the directory (often, this
  is the order in which they were created, but this depends on the
  details of the file system).}.

\Optiondesc{-r, --reverse} \index[REVERSE]{\Command{ls}: reverse
  order}Reverse the order while sorting.

\subsubject{\Command{locate} --- Look up files by name}
\index{searching+\Command{locate} --- look up files by name}
\Prompt{locate [OPTION]... PATTERN ...}  \Command{locate} takes one or
more patterns and looks for matching files in a periodically updated
database of file names.  The \Arg{PATTERN} arguments can be simple
text strings (which match any filename that contains them), shell
wildcard patterns, or --- when combined with the \Arg{--regex} option
--- regular expressions.

If you are looking for a certain file, \Command{locate} is a fast and
simple tool, with the following restrictions:
\startitemize
\item \Command{locate} can only find files which existed when the
  database of filenames was last updated.
\item Not all files are included in the database.  For example,
  personal directories of users are not indexed on the UAntwerpen HPC
  systems.
\stopitemize

Example:
\Prompt{locate */hosts}

\subsubject{\Command{find} --- Search for files in a directory hierarchy}
\index{searching+\Command{find} --- search for files} Like
\Command{locate}, \Command{find} can search for files, but it offers
many more options and works according to a different principle:
\startitemize
\item \Command{find} does not rely on databases of file names, but
  actively searches the file system while it is running instead.  Therefore,
  \Command{find} can see any file that exists at the time it is called
  (if you have reading permission for the directory where it is
  located).
\item Because \Command{find} searches the file system instead of a
  fixed database, it can be much slower than \Command{locate}.
\item \Command{find} can use arbitrary combinations of many different
  search criteria besides the file name, including but not limited to
  file ownership, permissions, creation, access or modification times.
\item Using \quotation{actions} such as \Arg{-exec} or \Arg{-ok}, you
  can immediately run arbitrary commands on the matched files.
\stopitemize
Because of its many features, \Command{find} is a powerful, but
complex, command.  It is called according to the following convention:
\Prompt{find [PATH...] [EXPRESSION]}

\Command{find} recursively searches for files that match the search
criteria in \Arg{EXPRESSION}, starting from every location in the list
of \Arg{PATH} arguments.  The list of search \Arg{PATH}s must appear
first, the \Arg{EXPRESSION} must appear last.  Arguments are optional:
without \Arg{PATH}, \Command{find} searches from the current working
directory; without \Arg{EXPRESSION}, \Command{find} matches every file
it encounters, and prints the file name.

Search criteria and actions are given in \Arg{EXPRESSION}.  Many
properties of a file can be used as search criteria, but we only list
a few of the most common {\em tests}.  The manual for \Command{find}
has a complete list.

\Optiondesc{-name PATTERN} Test if the file name matches
\Arg{PATTERN}, a shell wildcard expression.  Example: \Arg{-name
  '*.txt'} matches files ending in \Arg{.txt}.

\Optiondesc{-type [dfl...]} Test if the file is a \Arg{d}irectory,
regular \Arg{f}ile or symbolic \Arg{l}ink.  Refer to the manual for
more exotic file types.

\Optiondesc{-size n[bcwkMG]} Test the size of the file, in units of
\Arg{b}locks, bytes (\Arg{c}), \Arg{w}ords, \Arg{k}ilobytes,
\Arg{M}egabytes or \Arg{G}igabytes.  To find files bigger than or
smaller than a given size, use a prefix of \Arg{+} or \Arg{-}
respectively. Example: \Arg{-size +1M} matches files larger than 1
Megabyte.

\Optiondesc{-readable, -writable, -executable} Test if you can read,
write or execute the file.

If an expression contains multiple tests in a row, files must satisfy
all of them to be considered a match.  Alternatively, you can combine
tests with \Arg{-or}, or negate the result of a test with \Arg{'!'} or
\Arg{-not}.  For complex combinations of tests, you may need to use
parentheses \Arg{'('} \Arg{')'} to enforce the correct precedence.

\Arg{EXPRESSION} can also contain {\em actions} to be performed on
matching files.  If no action is given, \Command{find} prints the name
of each matching file.

\Optiondesc{-exec COMMAND \{\} ';'} Run an arbitrary \Arg{COMMAND} on
the matching files.  The name of each matching file is inserted in
place of the \Arg{\{\}} braces.  \quote{\Arg{;}} indicates the end of the
command (you should quote or escape it, because the shell will
interpret an unquoted \Arg{;} as the end of your \Command{find}
command, instead of passing it on to \Command{find}).

\Optiondesc{-ok COMMAND \{\} ';'} Identical to \Arg{-exec}, but asks
for confirmation before executing a \Arg{COMMAND}.

Example: find all files in the present directory and its subdirectories:

\Prompt{find}

Example: find all pictures (with extension {\tt .jpg}) under the
present directory:

\Prompt{find . -name '*.jpg'}

Note the \Arg{''} quotes surrounding the pattern {\tt
  *.jpg}\footnote{Without quotes, the shell could expand {\tt *.jpg}
  into a list of file names instead of passing it on as an argument to
  \Command{find}.}.  The \quote{\tt .} is optional here, as
\Command{find} searches the present directory by default if no
\Arg{PATH} arguments are given.

We can add an action to the command to move the matching files to
another location: \Prompt{find . -name '*.jpg' -exec mv {}
  \$HOME/Pictures/ ';'} %$

We can use \Arg{-or} and parentheses to include files with extension
{\tt .png} as well:

\Prompt{find . '(' -name '*.jpg' -or -name '*.png' ')' \backslash \\
    -exec mv \{\} \$HOME/Pictures/ ';'}

\index{archive}\index{compression}
\subsubject{\Command{tar} --- Create or unpack tar archives}
\Prompt{tar [OPTION]... [FILE]...}
\index[TAR]{\Command{tar}} \Command{tar} --- originally an
abbreviation for \quotation{tape archive} --- can create and extract
{\tt .tar} archive files.  \Command{tar} files can contain arbitrary
sets of files and directories, including their ownership and
permission data.  Plain \Command{tar} files do not compress the data,
therefore \Command{tar} is usually combined with \Command{gzip},
\Command{xzip} or \Command{bzip} to compress the archive file.

First, choose one of the following options to tell \Command{tar} to
\Arg{c}reate, e\Arg{x}tract or list the contents of an archive, and
which file to work with:

\Optiondesc{-c} Create an archive file.

\Optiondesc{-x} Extract the contents of an archive file.

\Optiondesc{-t} List the contents of the archive.

\Optiondesc{-f FILE} Provides the name of the archive file you want to
create, extract or list.  When you create a new archive, and a file
with the same name already exists, it is replaced.  If no \Arg{-f}
argument is given, GNU \Command{tar} reads archive files from
standard input, or writes files to standard output.

\Optiondesc{-v} Enable verbose output.  If you use \Arg{-v} together
with options \Arg{-c} or \Arg{-x}, \Command{tar} prints the name of
every file it archives or extracts.  If you combine \Arg{-v} with
option \Arg{-t}, \Command{tar} prints a verbose listing of the archive
contents, showing ownership and permissions, size and modification
date.

To use compressed archive files, provide one of the following options:

\Optiondesc{-z} \Command{gzip}

\Optiondesc{-J} \Command{xzip}

\Optiondesc{-j} \Command{bzip}

Apart from a number of options, \Command{tar} accepts a list of
\Arg{FILE} arguments, which can be files or directories.  When you
create an archive (\Arg{-c}), this is the list of files and/or
directories that should be included.  When you extract or list the
contents of an archive, the \Arg{FILE} arguments are optional:  if a list is given,
only those files or directories will be extracted or listed; if no
list is given, \Command{tar} extracts or lists all of the archive's
contents.

\subsubsubject{Examples}

Extract the contents of a file:

\Prompt{tar -xf archive.tar.gz}

Archive all {\tt txt} files in the current directory, with
\Command{gzip} compression:

\Prompt{tar -czf textfiles.tar.gz *.txt}

List the contents of the previously created file:

\Prompt{tar -tf textfiles.tar.gz}

Archive the contents of a directory, with \Command{xzip} compression:

\Prompt{tar -cJf my_directory.tar.zs ./my_directory}

\subsubject{\Command{zip} --- Create and modify zip archives}

\Prompt{zip [OPTION]... [ZIPFILE] [FILES]...}
\index{zip files+\Command{zip}: compress files} The \Command{zip} commands
creates archive files in the familiar .zip format.  At the command
line, specify options first, followed by the name of the .zip archive
you want to create or modify, and the list of \Command{FILE}s you want
to include in the archive.  If the provided .zip archive already
exists, \Command{zip} will add the \Command{FILE}s to the existing
archive, or update them if the archive already contains files with the
same name.

\Optiondesc{-r, --recurse-paths} The list of \Command{FILE}s to be
added to the archive can contain directory names as well.  By default,
the resulting .zip archive will only contain the (empty) directory.
Use the option \Arg{-r}, to add the contents of these directories
--- and their subdirectories --- as well.

\Optiondesc{-0, -1, -2, ..., -9} A number from \Arg{-0} to
\Arg{-9} sets the compression level.  The default level is
\Arg{-6}.  Higher levels will result in smaller archive files, but
might take more time to compress or extract.

We refer to the command's man page for many more advanced options to
add files to an archive, update or remove existing files, or select
files using on pattern matching.

\subsubsubject{Examples}

Compress all {\tt txt} files in the current directory:

\Prompt{zip txtfiles.zip *.txt}

Compress a directory with all contents:

\Prompt{zip -r my_directory.zip ./my_directory}

\subsubject{\Command{unzip} --- Extract files from zip archives}

\Prompt{unzip [OPTION]... [ZIPFILE] [FILES]...}

\index{zip files+\Command{unzip}: extract files} Use \Command{unzip}
to extract contents from .zip archives.  To extract all contents of
the archive \Command{myfile.zip} into the current working directory,
simply call \Command{unzip} with a single argument:

\Prompt{unzip myfile.zip}

Optionally, you can provide a list of \Command{FILE}s to extract only those
files from the archive.

To inspect the contents of an archive {\em without extracting} them, you can
use the options \Command{-l} or \Command{-v}.

\Optiondesc{-l} List archive contents (short format): print names,
uncompressed file sizes and modification dates.

\Optiondesc{-v} List archive contents (verbose format): print
additional information about the files contained in the archive, such
as their compressed file size and the compression method used.

\chapter{Processes}
\subsubject{\Command{ps} --- List processes}
\Prompt{ps [OPTIONS]}\index[PS]{\Command{ps} --- List
  processes}\Command{ps} provides information on processes currently
running on the system.  Without any options, only processes running in
your current shell session are listed.  For each process, \Command{ps}
prints the process id ({\tt PID}), name of the {\tt COMMAND} and used
processor {\tt TIME}, as well as the {\tt TTY} interface\footnote{{\tt
    tty} originally stands for TeleTYpewriter, and is the Linux system
  interface for keyboard terminal sessions.} it is connected to.

Using various options, you can control which information is shown, and
which processes are listed.  Note that, for historical reasons,
\Command{ps} accepts options with and without leading \quote{\Arg{-}}.

\Optiondesc{x} Also show processes not connected to a terminal. 

\Optiondesc{a} Also show processes from other users.  If combined with
\Arg{x}, \Command{ps ax} shows all processes by all users.

\Optiondesc{u} displays processes in \quotation{user-oriented} format:
displays some additional information, such as user name, processor and
memory usage, and the full command line that started the process.

\Optiondesc{o format} display information according to a chosen
\Arg{format}, a list of names of fields such as \Arg{pid}, \Arg{user},
\Arg{command}, \textellipsis\ See the \Command{man} page for
\Command{ps} for a full description of all formatting fields and
options.

\Optiondesc{k keys} \index{sorting+\Command{ps}: sort output}sort
output according to the given list of \Arg{keys}.  \Arg{keys} is a
comma-separated list of fields, and the output is hierarchically
sorted according to the chosen keys.  For example, the command
\Command{ps au --sort user,pid} sorts processes by user id, and then
sorts processes for each user by their process id.  Add \quote{\tt -}
in front of the field name to sort in reverse order.  Again, we refer
to the \Command{man} page for a full list of sorting options.

\Optiondesc{U userlist} Given the option \Arg{U} and an argument
\Arg{userlist}, a comma-separated list of one or more user names,
\Command{ps} only displays processors that belong to one of those
users

\Optiondesc{-T} Display a separate line of information for each of a
process' threads.

\subsubsubject{Examples}

List all processes, in user-oriented format:
\Prompt{ps aux}

List all processes, sorted by ascending CPU usage:
\Prompt{ps aux k\%cpu}

\subsubject{\Command{top} and \Command{htop} --- List processes interactively}
\index[TOP]{\Command{top} --- List processes interactively}
\index[HTOP]{\Command{htop} --- List processes interactively}
\Command{top} --- and its younger cousin \Command{htop}
--- can be regarded as an interactive, \quotation{live} version of
\Command{ps}.
\Prompt{top [OPTIONS]}

Usually, you run \Command{top} without any options to see a
continuously updated list of processes, and some system usage
statistics.  By default, processes are listed in order of decreasing
processor usage (\quote{{\tt CPU}}).  Not all processes will fit in the terminal,
but you can use the arrow keys to scroll up or down.  \Command{top}
can also show the threads of each process if you switch to thread view
by pressing {\bold\tt H}.

Some of the fields shown by default are:

\Descr{\tt PID} The process id.

\Descr{\tt USER} The user id.

\Descr{\tt RES} Resident memory, a good measure for the amount of physical
memory currently occupied by the process.

\Descr{\tt SHR} Shared memory, the amount of resident memory that is
shared with other processes.

\Descr{\tt \%CPU} The percentage of available processor time currently
used by the process, measured in percents of one processor core ---
for multi-threaded processes, this can be more than 100\%.

\Descr{\tt \%MEM} Memory usage: resident memory as a percentage of
available memory.

\Descr{\tt TIME} Total processor time used by the process so far.

\Command{top} doesn't show all available information by default.  To
show further data (or disable fields that don't interest you), press
{\bold\tt f} (\quotation{Fields Management}).  This takes you to a
menu with a list of all available fields, where you can select new
fields using the arrow keys, and mark them for display with the
{\bold\tt d} key.  A very useful field is {\bold\tt P}, the processor
core a thread or process last ran on.  This allows you to verify
whether multiple threads and processes are running on the same core,
which can slow down parallel calculations, sometimes by a big factor.

\Command{htop} is a more modern version of \Command{top} which
presents similar information in a slightly more advanced interface.
It displays usage bars for every processor core on the system, as well
as the memory, which gives you a quick impression of the system's
total load.  Hit function key {\tt\bold F1} or {\tt\bold h}, to get
help; {\tt\bold F2} or {\tt\bold C} to configure which fields are
displayed.

\subsubject{\Command{jobs} --- Controlling jobs in the shell}
\Prompt{jobs}
\index[JOBS]{\Command{jobs} --- list active jobs in shell}Every active
pipeline (consisting of one or more processes) in the current shell
session is known as a \quote{job}.  If you move jobs to the
background, either by terminating a pipeline with \Arg{\&}, or by
interrupting them with {\tt\bold Ctrl + z}, multiple
active jobs can exist at the same time.

Jobs are numbered starting from {\tt 1}.  The \Command{jobs} command
prints a list of all currently active jobs, with their job number,
status (\quotation{Running} or \quotation{Stopped}) and the pipeline
used to start them.

Commands that act on jobs, such as \Command{kill}, \Command{fg} (run
job in foreground\index[FG]{\Command{fg} --- run a job in the
  foreground}) and \Command{bg} (run job in the
background\index[BG]{\Command{bg} --- run a job in the background}),
identify jobs by a \Arg{jobspec}\index[JOBSPEC]{{\tt jobspec}}
argument.  This is the number of the job as reported by
\Command{jobs}, preceded by {\tt\%}.  For example, the following
command lets job {\tt 2} continue running in the background:

\Prompt{bg \%2}

\subsubject{\Command{kill} --- Terminate or pause processes}

\Prompt{kill [-sigspec] pid \| jobspec ...}

\index[KILL]{\Command{kill} --- terminate or pause processes}
\Command{kill} can terminate or interrupt processes by sending them a
signal.  You can terminate a single process by providing its
\Arg{pid}, the process id which you can obtain from \Command{ps} or
\Command{top}.  For example, the following command terminates process {\tt 54399}:

\Prompt{kill 54399}

You can also terminate a shell job instead of a simple process, by
providing a \Arg{jobspec} instead of a \Arg{pid}.  If this job is a
pipeline consisting of multiple processes, all those processes are
terminated by the same \Command{kill} command.  For example, to
terminate job {\tt 2}, use the following command:

\Prompt{kill \%2}

By default, \Command{kill} sends processes the \Arg{TERM} signal,
asking them to clean up and exit.  However, there are other signals
you can send, using the optional argument \Arg{-sigspec}.  There are a
number of different signals, each of which has a different meaning.
Every signal has a name and a number which can be used
interchangeably.  For example, the \Arg{STOP} signal has number
\Arg{19}, so the following commands are equivalent:

\Prompt{kill -STOP 54399}

\Prompt{kill -19 54399}

We list a few of the most common signals, you can run \Command{kill
  -l} to see a full list.

\Optiondesc{-TERM, -15} Processes receiving a \Arg{TERM} signal are
not immediately stopped, but get a chance to shut down gracefully (for
example by saving intermediate data).  This is the clean way to stop a
process, and also the signal sent by default if you do not provide a
\Arg{-sigspec} argument.

\Optiondesc{-KILL, -9} The \Arg{KILL} signal immediately ends the
process.  If you want to end a process, the recommended way is to try
the \Arg{TERM} signal first (i.e.\ use \Command{kill} without
\Arg{-sigspec} argument), and try \Arg{-KILL} if a process does not
respond to \Arg{TERM}.

\Optiondesc{-STOP, -19} The \Arg{STOP} signal stops (pauses) the
process.  All calculations are stopped, but the process is not
destroyed and may be resumed later.

\Optiondesc{-CONT, -18} The \Arg{CONT} signal continues a previously
stopped process.

\subsubject{\Command{time} --- Measure the time used by a command}
\Prompt{time COMMAND}\index[TIME]{\Command{time} --- measure running
  time and CPU use}If you precede a command (or pipeline) by
\quote{\Command{time}}, the shell will report the total time taken by
that command.  \Command{time} reports three values:

\Descr{real} \quotation{real} time is the actual time that has passed
in the physical world (sometimes referred to as \quotation{walltime}
--- the time that has gone by as measured by a clock on the wall).

\Descr{user} This is the time spent by processor cores (\quotation{CPU
  time}) executing the code of the command itself.  If the command
uses multiple processor cores, this value can be larger than the real
time.  Conversely, user time can be much less than real time if the
command doesn't do a lot of work (for an example, try \Command{time
  sleep 5}).

\Descr{sys} The CPU time spent by the operating system (kernel) on
behalf of your command.  Examples of tasks that count towards
\quotation{system} time are: input/output, thread management and
allocation of memory.

\chapter{Working with text}
\subsubject{\Command{wc} --- count lines, words or characters}
\Prompt{wc [OPTION]... [FILE]...}  \index[WC]{\Command{wc} --- count
  lines, words or characters}\Command{wc} prints the number of lines,
words and bytes (or characters) for each file, and -- if more than one
file is given --- the total counts.  If no file is given, \Command{wc}
reads from standard input.  You can select what is printed by
providing one or more of the following options:

\Optiondesc{-l, --lines}{}

\Optiondesc{-w, --words}{}

\Optiondesc{-c, --bytes}{}

\Optiondesc{-m, --chars}{}

\subsubject{\Command{less} --- Scroll through text}
\index[LESS]{\Command{less} --- scroll through text} \Prompt{less
  [FILE]} \Command{less} displays the content of a file, or --- if no
\Arg{FILE} is provided --- standard input.

While \Command{less} is running, you can navigate the text by pressing
the appropriate keys.  Some useful commands:

\Optiondesc{j} Scroll forward one line (you can also use the down arrow key).

\Optiondesc{k} Scroll backward one line (you can also use the up arrow key).

\Optiondesc{f} Scroll forward one window.

\Optiondesc{b} Scroll backward one window.

\Optiondesc{q} Quit.

\Optiondesc{h} Read help, where you can learn about further commands
not explained here.  Press {\tt\bold q} once to leave the help screen
and go back to the text screen.

A very useful feature is the {\em search} function.  To search for a word in
the text you are viewing:
\startitemize[n]

\item Press the {\bold\tt /} (\quotation{slash}) key.  A cursor appears at the
  bottom of the terminal.

\item Type the word you want so search for, and press {\bold\tt enter}.
  \Command{less} jumps ahead to the next occurrence of that word (if
  it appears anywhere below your current position in the text).

\item Press {\bold\tt n} to go to the next occurrence, press {\bold\tt
  Shift + n} to go back to the previous occurrence.

\stopitemize
\subsubject{\Command{head} --- \index[START]{print the start}Print the
  first part of files}

\Prompt{head [OPTION]... [FILE]...}
\index[HEAD]{\Command{head}}\Command{head} prints the first 10 lines
of each \Arg{FILE} on standard output.  If no \Arg{FILE} arguments are
provided (or when \Arg{FILE} is \Arg{-}), \Command{head} reads from
standard input instead.  Using the option \Arg{-n} or \Arg{--lines=}
with an integer value \Arg{k}, you change the number of printed lines.
With a negative number \Arg{--lines=-k}, \Command{head} prints all but
the last \Arg{k} lines.

Example: print the first 5 files with the extension {\tt .txt} in the
current directory.  \Prompt{ls *.txt \| head -n 5 }

\subsubject{\Command{tail} --- \index[END]{print the end}Print the last
  part of files}

\Prompt{tail [OPTION]... [FILE]...}
\index[TAIL]{\Command{tail}}\Command{tail} prints the last 10 lines of
each \Arg{FILE} on standard output (it is thus a mirror version of
\Command{head}).  If no \Arg{FILE} arguments are provided (or when
\Arg{FILE} is \Arg{-}), \Command{tail} reads standard input.  Using
the option \Arg{-n} or \Arg{--lines=} with an integer value \Arg{k},
print the last \Arg{k} lines.  If you provide the option
\Arg{--lines=+k}, \Command{head} prints everything starting from the
\Arg{k}th line.

\subsubject{\Command{cat} --- Concatenate files and print}

\Prompt{cat [OPTION]... [FILE]...}
\index[CAT]{\Command{cat}}\Command{cat} concatenates the \Arg{FILE}s
passed as arguments, and prints the result on standard output.
\Command{cat} with a single \Arg{FILE} argument is a quick way to
print the entire content of that file on standard output:
\Prompt{cat file1}

Often, \Command{cat} is used with an output redirection to combine a
number of files into a new file:

\Prompt{cat file1 file2 > file_combined.txt}

Without \Arg{FILE} arguments, or with the argument \quote{\Arg{-}},
\Command{cat} reads from standard input.  This can be used to combine
the content of existing files with the output of a command.  For
example, the following pipeline inserts the output of \Command{ls}
between two existing files {\tt prologue.txt} and {\tt epilogue.txt}:

\Prompt{ls \| cat prologue.txt - epilogue.txt > my_directory.txt}

\Command{cat} provides a number of options to decorate the output in
various ways, e.g.\ by adding line numbers, for which we refer to the
\Command{man} page.

\subsubject{\Command{sort}}
\index{sorting+\Command{sort} command}
\Prompt{sort [OPTION]... [FILE]...}

The \Command{sort} command sorts and then prints the lines of all
input files (or standard input).  By default, lines are sorted
alphabetically, but other options are available.

\Optiondesc{-r, --reverse} Reverse the sorting order.

\Optiondesc{-n, --numeric-sort} Sort according to the numerical value.

\Optiondesc{-f, --ignore-case} Treat lower case characters as upper
case when comparing.

\Command{sort} can also work with tabular data, i.e.\ text data where
each line consists of different fields.  With the option \Arg{-k
  f1,f2} (or \Arg{--key=f1,f2}), for numbers \Arg{f1} and \Arg{f2},
\Command{sort} uses the contents of fields \Arg{f1} up to \Arg{f2} to
sort the lines (\Arg{f2} can be omitted, in which case the contents of
all fields starting from \Arg{f1} up to the end of the line is used).  We
refer to the official documentation for more complex key
specifications and options.

By default, fields are separated by space or tab characters.  With the
option \Arg{-t SEP} (or \Arg{--field-separator=SEP}), \Command{sort}
uses the character \Arg{SEP} to delimit the fields.

Example: an alternative way to sort the output of \Command{ls} by file
size, using \Command{sort} on numeric value of the fifth field:
\Prompt{ls -l \| sort -k 5,5 -n} This produces similar\footnote{Small
  differences arise from the combination of alphabetical and numerical
  sorting. To exactly reproduce the output of \Command{ls}'s built-in
  \Arg{--sort} option, use the following slightly more complex
  \Command{sort} pipeline: \Command{ls -l \| sort -k 5n,5 -k 9r,9}}
output to the following command, which uses the built-in \Arg{--sort}
option of \Command{ls}: \Prompt{ls -l --sort=size --reverse}.

\subsubject{\Command{uniq} --- Detect repeated lines}
\index[UNIQ]{\Command{uniq} --- detect repeated lines}

\Prompt{uniq [OPTION]... [INPUT [OUTPUT]]}

\Command{uniq} reads lines from the file \quote{\Arg{INPUT}}, and
prints to the file \quote{\Arg{OUTPUT}}.  If two or more consecutive
lines are identical, only one repetition is printed.  \Arg{INPUT} or
\Arg{OUTPUT} are optional arguments, and may be replaced by standard
input or output instead.  Some extra options are:

\Optiondesc{-c} Print the number of repetitions of each line.

\Optiondesc{-d} Only print duplicated lines.

\Optiondesc{-u} Only print those lines which are not repeated.

\Optiondesc{-i} Ignore case when comparing lines.

Note that \Command{uniq} only detects {\em consecutive} identical
lines.  Therefore, a classic technique is to first move identical
lines together using \Command{sort}, and then filter the output with
\Command{uniq}, as in the following example:

\Prompt{sort somefile.txt \| uniq}

or, if you want to see unique lines of the output of a command:

\Prompt{COMMAND \| sort \| uniq}

\subsubject{\Command{cut} --- Extract parts of text lines}
\index[CUT]{\Command{cut} --- extract parts of text
  lines}\Command{cut} is a useful tool to extract parts of structured
text.

\Prompt{cut OPTION... [FILE]...}

\Command{cut} takes its input from (a list of) files, or standard
input.  Depending on the chosen \Arg{OPTION}, \Command{cut} extracts
either fixed ranges of characters, or fields:

\Optiondesc{-c LIST}Print the character ranges specified in \Arg{LIST}.

\Optiondesc{-f LIST}Treat lines of text as table rows with different
fields, and print the fields from the given \Arg{LIST}.  By default,
the TAB character marks the end of a field, but you can choose a
different field separator using the option \Arg{-d}.

\Optiondesc{-d CHAR} In combination with the \Arg{-f} option: set the
field separator to the right character of choice, if fields are not
separated by TAB.

The format of the \Arg{LIST} argument for options \Arg{-c} and
\Arg{-f} is the same: a list of one or more integers \Arg{i} or
integer ranges \Arg{i-j}, separated by commas.  For example, the
following command prints characters 1, 3, 4 and 5 of each line of the
file {\tt input.txt}:

\Prompt{cut -c 1,3-5 input.txt}

The following command prints columns 3 and 5 of a file with columns
separated by a single space:

\Prompt{cut -d ' ' -f 3,5 columns.txt}

\subsubject{\Command{grep} --- print matching lines}

\Prompt{grep [OPTION]... PATTERN [FILE...]}
\index[GREP]{\Command{grep}}\Command{grep} searches given input files
(or standard input if no files are given) for a pattern and prints the
matching lines.  By default \Arg{PATTERN} is interpreted as a basic
regular expression, the option \Arg{-E} enables extended regular
expressions.  Typical uses are
\startitemize
\item searching\index{searching+\Command{grep} --- search inside
  files} for files that contain certain words, or
\item filtering the output of commands (or long files) to print only
  lines containing specific information.
\stopitemize

\Command{grep} provides many options to configure what is matched and
what gets printed.  Some of the most commonly used options
are:

\Optiondesc{-E, --extended-regexp} Interpret \Arg{PATTERN} as an
extended regular expression.

\Optiondesc{-i, --ignore-case} Allow matches with characters that have
a different (upper/lower) case.

\Optiondesc{-v, --invert-match} Select those lines which {\em do not}
match the pattern.

\Optiondesc{-r, --recursive} Recursively search all files under each
subdirectory, if one or more of the \Arg{FILE} arguments is a
directory.

Example: list files in the current directory which were last modified
in March or April.

\Prompt{ls -l \| grep -E 'Mar\|Apr'}

It is often a good idea to enclose the regular expression
\Arg{PATTERN} in single quotes.  Otherwise, the shell might interpret
special characters (such as parentheses, the asterisk or the pipe
symbol), instead of passing the literal \Arg{PATTERN} to
\Command{grep}.

\chapter{Variables}
\subsubject{\Command{read} --- Read variables from standard input}
\index[READ]{\Command{read} --- variables from standard input}
\Prompt{read [NAME ...]}  The built-in shell command \Command{read}
takes a list of variable \Arg{NAME}s as arguments.  It reads a line of
text from standard input, splits it into a list of words, and assigns
each word to the corresponding variable (if the line contains more
words than there are variable arguments, the entire remaining part of
the input line is assigned to the last variable name).

\Command{read} can be used to get user input in interactive scripts,
but it can also be used to process text output.  A common idiom is to
combine \Command{read} with a \Command{while} loop. For example, the
following code successively assigns each line of output from
\Command{ls -la} to the variable {\tt LINE}, and runs further commands
to process each line.

\startSHELL
  ls -la | while read LINE
  do
      # sequence of commands that work on $LINE
      ...
  done
\stopSHELL%$

\subsubject{\Command{export} --- Create environment variables}

\Prompt{export name[=value]}

\index[EXPORT]{\Command{export} --- Create environment variables}The
\Command{export} command creates an environment variable with the
given \Arg{name}.  If \Arg{name} is the name of an existing shell
variable, its value is preserved.  To create the variable and set its
value with a single command, include the optional \Arg{=value}.

\subsubject{\Command{set} --- List shell variables}
\index[SET]{\Command{set} --- list shell variables}If you run
\Command{set} without any further arguments, it prints a list of all
variables and functions defined in the current shell.  \Command{set}
can also change various options that affect the behaviour of the
shell.  We refer to the Bash manual for these advanced options.

\subsubject{\Command{printenv} --- List environment variables}
\index[PRINTENV]{\Command{printenv} --- list environment variables} The
command \Command{printenv} prints all current environment variables
and their values.

\completeindex

\stoptext
%%% Local Variables:
%%% mode: context
%%% TeX-master: t
%%% End:
